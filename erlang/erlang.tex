\subsection{Variables, Atoms, Tuples, Lists}
\textbf{Variables} start with an Upper Case Letter (like in Prolog).
Variables can only be bound once, the value of a variable can never be changed once it has been set.\\
\textbf{Atoms} are like symbols in Scheme.
Any character code is allowed within an atom, singly quoted sequences of characters are atoms (not strings).
unquoted must be lowercase, to avoid clashes with variables.\\
\textbf{Tuples} are used to store a fixed number of items.\\
Are like in Haskell, e.g. [1, 2, 3], ++ concatenates.
Main difference: [X | L] is cons (like (cons X L)).
Strings are lists, like in Haskell.
\begin{lstlisting}
	ACamelCaseVariableName % Variable (unmutable)
	start_with_a_lower_case_letter % Atom (symbol)
	{person, 'Jim', 'Austrian'} % Tuple: three atoms
\end{lstlisting}

\subsection{Pattern Matching}
Like in Prolog, = is for pattern matching; \_ is “don’t care”.
\begin{lstlisting}
	A = 10 % Succeeds, binds A to 10
	{A, A, B} = {abc, abc, foo} % Succeeds - binds A to abc, B to foo
	{A, A, B} = {abc, def, 123} % Fails
	[A,B|C] = [1,2,3,4,5,6,7] % Succeeds - binds A = 1, B = 2, C = [3,4,5,6,7]
	[H|T] = [abc] % Succeeds - binds H = abc, T = []
	{A,_, [B|_], {B}} = {abc, 23, [22, x], {22}} % 	Succeeds - binds A = abc, B = 22
\end{lstlisting}

\subsection{Maps}
\begin{lstlisting}
	Map = #{one => 1, "Two" => 2, 3 => three}. % Create
	Map#{one := "I"}. % Update/insert
	#{"Two" := V} = Map. % Search and assign
\end{lstlisting}

\subsection{Functions and Modules}
Function and \textbf{Module} names (func and module in the above) must be atoms.
Functions are defined within Modules.
Functions must be exported before they can be called from outside the module where they are defined.
\begin{lstlisting}
	-module(demo).
	-export([double/1]).
	double(X) -> times(X, 2). % Exported
	times(X, N) -> X * N. % Not exported
\end{lstlisting}
Built-in functions in the erlang module:
\begin{lstlisting}
	date()
	time()
	length([1,2,3,4,5])
	size({a,b,c})
	atom_to_list(an_atom) % "an_atom"
	list_to_tuple([1,2,3,4]) % {1,2,3,4}
	integer_to_list(2234) % "2234"
	tuple_to_list({1,2,3,4}) % [1,2,3,4]
\end{lstlisting}
A \textbf{Function} is defined as a sequence of clauses (separated by ";").
\textbf{Clauses} are scanned sequentially until a match is found.
When a match is found all variables occurring in the head become bound.
Variables are local to each clause, and are allocated and deallocated automatically.
The body is evaluated sequentially (use "," as separator).
The keyword \texttt{when} introduces a guard, like "$\mid$" in Haskell.
All variables in a guard must be bound.
\begin{lstlisting}
	factorial(0) -> 1;
	factorial(N) when N > 0 ->
		N * factorial(N - 1).
\end{lstlisting}

\subsection{Apply}
\begin{lstlisting}
	apply(Mod, Func, Args)
\end{lstlisting}
Apply function Func in module Mod to the arguments in the list Args.
Mod and Func must be atoms (or expressions which evaluate to atoms).
Any Erlang expression can be used in the arguments to apply.
\texttt{?MODULE} uses the preprocessor to get the current module’s name.

\subsection{Case and If}
Note that if needs guards, so for user defined predicates it is customary to use case.
\begin{lstlisting}
	case lists:member(a, X) of
		true -> ... ;
		false -> ...
	end,
	%%%%%
	if
		integer(X) -> ... ;
		tuple(X) -> ... ;
		true -> ... % works as an "else"
	end,
\end{lstlisting}

\subsection{Lambdas}
\begin{lstlisting}
	Square = fun (X) -> X*X end. % Declaration
	Square(3). % Usage
	lists:map(Square, [1,2,3]). returns [1,4,9] % Lambdas can be passed as usual to higher order functions:
	lists:foldr(fun my_function/2, 0, [1,2,3]). % To pass standard functions, we need to prefix their name	with fun and state its arity:
\end{lstlisting}

\subsection{Concurrent Programming: the Actor Model}
Everything is an \textbf{Actor}: an independent unit of computation.
Actors are inherently concurrent.
Actors can only communicate through messages (async communication).
Actors can be created dynamically.
No requirement on the order of received messages.\\
There are three main primitives:
\begin{itemize}
	\item \texttt{spawn} creates a new process executing the specified function, returning an identifier.
	\item \texttt{send} (written \texttt{!}) sends a message to a process through its identifier; the content of the message is simply a variable. The operation is asynchronous.
	\item \texttt{receive} end extract, going from the first, a message from a process’s mailbox queue matching with the provided set of patterns, this is blocking if no message is in the mailbox. The mailbox is persistent until the process quits.
\end{itemize}

\subsection{Creating a new process}
We have a process with \texttt{Pid1} (\textbf{Process Identity} or Pid).
In it we perform \texttt{Pid2 = spawn(Mod, Func, Args)}.
Like apply but spawning a new process.
After, \texttt{Pid2} is the process identifier of the new process, this is known only to process \texttt{Pid1}.

\subsection{Message Passing}
\begin{lstlisting}
	PidB ! {self(), {mymessage, [1,2,3]}} % Process A sends a message to B
	receive
		{From, {Msg, Data}} ->
			work_on_data(Data);
	end
\end{lstlisting}
\texttt{self()} returns the Pid of the process executing it.
\texttt{From}, \texttt{Msg} and \texttt{Data} become bound when the message is received.
Messages can carry data and be selectively unpacked.
If \texttt{From} is bound before receiving a message, then only data from that process is accepted.\\
Client-Server can be easily realized through a simple protocol, where requests have the syntax \texttt{\{request, ...\}}, while replies are written as \texttt{\{reply, ...\}}

\subsection{Registered Processes}
\texttt{register(Alias, Pid)} registers the process Pid with name Alias.
Any process can send a message to a registered process.
\begin{lstlisting}
	start() ->
		Pid = spawn(?MODULE, server, [])
		register(analyzer, Pid).
	analyze(Seq) ->
		analyzer ! {self(), {analyze, Seq}},
		receive
			{analysis_result, R} -> R
		end.
\end{lstlisting}

\subsection{Timeouts}
\begin{lstlisting}
	% If the message foo is received within the time Time perform Actions1 otherwise perform Actions2.
	wait(Time, message) ->
		receive
			foo -> Actions1;
		after
			Time -> Actions2;
	sleep(T) -> % process suspends for T ms.
		receive
		after
			T -> true
		end.
	suspend() -> % process suspends indefinitely.
		receive
		after
			infinity -> true
		end.
	flush() -> % flushes the message buffer.
		receive
			Any -> flush()
		after
			0 -> true
		end.
\end{lstlisting}